% LaTeX rebuttal letter example. 
\documentclass[12pt]{article}
\usepackage[utf8]{inputenc}
%\usepackage{lipsum} % to generate some filler text
\usepackage{fullpage}
\usepackage{graphicx}
\usepackage{soul}
% import Eq and Section references from the main manuscript where needed
% \usepackage{xr}
% \externaldocument{manuscript}

% package needed for optional arguments
\usepackage{xifthen}
% define counters for reviewers and their points

\usepackage{hyperref}

\newcounter{reviewer}
\setcounter{reviewer}{0}
\newcounter{point}[reviewer]
\setcounter{point}{0}

\usepackage{xcolor}
\definecolor{review-color}{rgb}{0.1, 0.1, 0.8}

\usepackage[english]{babel}
\usepackage{amsmath}

\makeatletter
\renewcommand{\thefigure}{A-\thereviewer.\@arabic\c@figure}
\renewcommand{\thetable}{A-\thereviewer.\@arabic\c@table}
\renewcommand{\thepage}{A-\arabic{page}}

\makeatletter


\addto\captionsenglish{% Replace "english" with the language you use
	\renewcommand{\contentsname}%
	{{\large Reviewer \hfill  Page}}%
}
\setcounter{secnumdepth}{0}
\setcounter{tocdepth}{1}

% This refines the format of how the reviewer/point reference will appear.
\renewcommand{\thepoint}{P\,\thereviewer.\arabic{point}} 

% command declarations for reviewer points and our responses
\newcommand{\reviewersection}{\stepcounter{reviewer} \bigskip \hrule
                  \section{Reviewer \thereviewer}}


\newenvironment{point}
   {\refstepcounter{point} \bigskip \noindent {\textbf{\color{blue}{Reviewer~Point~\thepoint} }} ---\ \color{review-color} }
   {\par }

\newcommand{\shortpoint}[1]{\refstepcounter{point}  \bigskip \noindent 
	{\textbf{\color{blue}{Reviewer~Point~\thepoint} }} ---~\textcolor{review-color}{#1}\par }

\newenvironment{reply}
   {\medskip \noindent \begin{sf}\textbf{\color{red}{Reply}}:\  }
   {\medskip \end{sf}}

\newcommand{\shortreply}[2][]{\medskip \noindent \begin{sf}\textbf{\color{red}{Reply}}:\  #2
	\ifthenelse{\equal{#1}{}}{}{ \hfill \footnotesize (#1)}%
	\medskip \end{sf}}



\begin{document}
%\section*{Cover Letter}
\begin{sf}
{\small 
\noindent
\textbf{\underline{Paper ID:}} TCAD-2020-0486 \\
\textbf{\underline{Title:}} HotCluster: A thermal-aware defect recovery method for Through-Silicon-Vias Towards Reliable 3-D ICs systems\\
\textbf{\underline{Authors:}} Khanh N. Dang, Akram Ben Ahmed, Abderazek Ben Abdallah, and Xuan-Tu Tran \\
%\textbf{\underline{Manuscript NO:}} TVLSI-00334-2019\\
\vspace{0.5cm}}
\begin{flushright}
	\today
\end{flushright}

\noindent
Dear Editors 

\vspace{.5cm}
\noindent
Thank you for allowing us to revise our manuscript, with an opportunity to address the reviewers’ 
comments. 
\vspace{.5cm}

\noindent
Please find attached our revised manuscript titled ``HotCluster: A thermal-aware defect recovery method for Through-Silicon-Vias Towards Reliable 3-D ICs systems'', which we previously submitted for publication as an original research article in IEEE Transactions on Computer-Aided Design of Integrated Circuits and Systems. The paper ID is TCAD-2020-0486.
\vspace{.5cm}

\noindent
We are uploading (a) our point-by-point response to the comments (below) (response to 
reviewers), (b) an updated manuscript with yellow highlighting indicating changes, and (c) a clean 
updated manuscript without highlights.
\vspace{.5cm}

\noindent
This work has not been submitted elsewhere. 

\vspace{.3cm}
\noindent
This work is based on our preliminary work in [21] with the additional new contributions as follows:


\begin{itemize}
	
	\item \colorbox{red}{Similar:} The architecture of TSV group as it is clustered into 4 subset around the router. The mapping approach is center priority (we called it CPWI).
	\item \colorbox{green}{New:} Architecture with redundant clusters of TSV. Work in [21] has no redundancy.
	\item \colorbox{green}{New:} A new online algorithm for  mapping named SAWI which is based on number of spare for each router.
	\item \colorbox{green}{New:} A new offline algorithm for  mapping based on max-flow min-cut theorem and solved with Ford-Fulkerson method.
	\item \colorbox{green}{New:} A thermal model for predicting fault rates that are accelerated by the operating temperature. The model is based on Arrherius's Law.
	\item \colorbox{green}{New:} A new algorithm for inserting redundant clusters and correcting instead of uniformly insertion like using SAWI, CPWI or Ford-Fulkerson approach.

	
\end{itemize} 

We believe the difference between this work the our preliminary work in [21] are substantial enough for publication. The evaluation results also show this work outperforms the previous work in [21].

The overlapping part with [21] is the section III.A where we present the preliminary work. Other parts are new. As we listed above, the new contribution is up to 5/6 ideas (83\%); therefore, we strongly believe the new paper is good new for publication and is not a self-plagiarism of [21]. If you still hesitate about the overlapping, we attached in the submission the paper [21] for your information.

\vspace{.3cm}

\noindent
{\small  \underline{Reference:}  [21]  Khanh N. Dang, Akram Ben Ahmed, Yuichi Okuyama, and Abderazek Ben Abdallah, ``Scalable design methodology and online algorithm 	for TSV-cluster defects recovery in highly reliable 3D-NoC systems,''
	IEEE Transactions on Emerging Topics in Computing, vol. 8, no. 3, 	pp. 577–590, 2020.
}

\vspace{.3cm}


\vspace{.3cm}
\noindent
The corresponding author is Khanh N. Dang (khanh.n.dang@vnu.edu.vn).

\vspace{.3cm}
\noindent
Thank you for considering our manuscript. 

\vspace{.5cm}


\noindent 
Respectfully yours, 

\vspace{.1cm}
\noindent
Khanh N. Dang 

\vspace{.2cm}



\noindent
SISLAB, VNU-UET, Vietnam National University Hanoi, Vietnam

\noindent
Tel: +84-904-899-490 

\noindent
E-mail:   khanh.n.dang@vnu.edu.vn 

\vspace{.3cm}
\noindent
On behalf of all authors.
\pagebreak
\end{sf}

\end{document}


